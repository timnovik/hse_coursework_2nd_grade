\begin{problem}{Ганзейский союз}{Входные данные}{Результат}{}

Историк Петя изучает Ганзейский союз - торговое объединение многих городов на севере Германии, существовавшее где-то между 13 и 15 веками.
В рамках своего исследования он читает дневники одного из торговцев этого союза, владевшего тогда целым кораблем и возившего с его помощью грузы между городами.

По словам торговца, в то время в союзе состояло целых $n$ городов, торговавших между собой $k$ различными товарами, каждый из которых стоил по-разному в зависимости от города.
Для каждого из $k$ товаров торговец записал, сколько золотых монет он отдавал за заполнение трюма своего корабля этим товаром в каждом городе.
Торговец был очень умным и потому, как всякий уважающий себя торговец в то время, неплохо наживался на разнице в стоимости товаров между городами: он заполнял трюм товаром в одном городе, плыл в другой и продавал, обычно - по более высокой цене.

Разумеется, не всё так просто, ведь на путешествия уходит много денег: выплаты матросам, закупка продовольствия, содержание корабля...
Впрочем, предприимчивый торговец и это учёл, и для каждой пары городов записал, во сколько обходится перемещение гружёного корабля между ними.
При этом торговец отметил, что если дорога от города $A$ в город $B$ обходится в $x$ золотых, то и обратная дорога - то есть от города $B$ в город $A$ - будет стоить те же самые $x$ золотых, а потому в своих дневниках торговец указал стоимость дороги только в одну сторону.

В своих заметках торговец жаловался на очень строгие правила союза: каждый корабль обязан был иметь на борту особый учётный лист, в котором описывался маршрут корабля и то, какой товар планировалось приобрести в каждом городе.
По этим же правилам длина учётного листа не должна была превышать $s$ городов, а в последнем городе нельзя было закупать товары: ведь в таком случае эти товары не попадали бы в ведомости, а ганзейские чиновники никак не могли такого допустить!

По словам торговца, очередное его путешествие началось в городе с номером $1$ и он планировал заработать как можно больше денег, пройдя по одному учётному листу.
К сожалению, в этом дневнике торговец описал только все города и товары, а вот описание его путешествия было утеряно. Петя для своего исследования хочет восстановить хотя бы возможный учётный лист.
Он не сомневается в том, что торговец, имея на руках все цифры, заполнил учётный лист с наибольшей выгодой для себя, но городов в союзе так много, что Петя не знает, как это сделать, и потому просит вас.

Пожалуйста, помогите ему найти подходящий учётный лист!

В первой строке входных данных находится число $t$ - количество тестовых наборов.

Каждый тестовый набор начинается со строки, содержащей три целых числа: $n, s, k$ - количество городов, максимальное количество городов в учётном листе и количество товаров соответственно.
Гарантируется, что $2 \le n \le 300, 2 \le s \le 200, 1 \le k \le 1000$.

Затем даётся $k$ строк, по $n$ чисел в каждой: $i$-ая строка обозначает стоимость товара номер $i$ в каждом из $n$ городов соответственно.

После этого даётся $n - 1$ строк с описаниями путей между городами.
В строке номер $i$ содержится $i$ целых чисел: стоимость дороги от городов с номерами с $1$ по $i$ до города $i+1$ соответственно.

Для каждого тестового набора вам необходимо один учётный лист, который позволит заработать наибольшее число золотых.
Формат вывода учётного листа следующий:

В первой строке выводится два числа $x, w$ - количество городов в маршруте и заработок на маршруте соответственно, $2 \le x \le s$.

Во второй строке выводится $x$ чисел от $1$ до $n$ - номера городов по порядку в маршруте. Первый город в каждом учётном листе - $1$.

В третьей строке находится $x - 1$ число от $1$ до $k$ - номера товаров, покупаемых по пути. $i$-ое число соответствует товару, покупаемому в $i$-ом городе по порядку следования в маршруте.

В случае, если не существует учётного листа, приносящего выгоду - необходимо вывести единственное число $-1$.



В первом тесте $t = 6, \ n \le 30, s \le 100, k \le 50$. Каждый тест оценивается в 5 баллов, проверка выполняется online.

Во втором тесте $t = 14$. Каждый тест оценивается в 5 баллов, во время тура проверяется лишь, что ответ соответствует формату вывода.

\Examples
\begin{example}
\exmp{5
2 2 2
1 2
2 4
1
3 3 2
1 4 1
2 1 8
2
4 1
4 3 2
1 7 2 4
4 2 3 1
4
2 1
3 2 2
2 4 2
1 3
3 1
1
2 2 1
2 1
2
}{2 1
1 2
2
3 7
1 2 3
1 2
3 3
1 3 2
1 1
4 3
1 2 1 2
1 2 1
-1}%
\end{example}

В первом тестовом наборе есть лишь два города и два доступных товара. Торговец может составить маршрут из двух городов, то есть единственный имеющийся у него вариант - пройти из города $1$ в город $2$. Стоимость этого перемещения - $1$ золотой. Если торговец купит в стартовом городе первый товар за $1$ золотой и продаст во втором городе за $2$ золотых - он выйдет в ноль. А вот если купит второй товар - получит $2$ золотых и итоговая прибыль составит $1$ золотой.

В третьем тестовом наборе оптимальный вариант - заработать на разнице в стоимости $1$-ого товара между городами $1$ и $2$.
Можно пройти напрямую, отдав за это $4$ золотых и получив доход в $2$ золотых, но более выгодно будет пройти транзитом через город $3$ и отдать за это $2 + 1 = 3$ золотых.
По правилам союза необходимо продать товар во втором городе, но сразу же после этого торговец выкупает его назад.

В четвёртом тестовом наборе одно путешествие между городами приносит $1$ золотой независимо от направления: при переходе $1 \rightarrow 2$ надо покупать первый товар, а при обратном переходе - второй товар.
Ограничения на длину листа позволяют совершить $3$ перехода и заработать $3$ золотых.

В пятом тесте единственный возможный путь убыточен, поэтому выводится $-1$.

\end{problem}
